\centerline{\bf\zihao{3}\onelinetitle}\phantomsection
\section{文类简介、选项与设置}
\subsection{文类简介}
\textbf{本文类是在TeX Live 2021环境下编写的,需要采用xelatex进行编译,虽然是TeX Live 2021版本,但老版本的TeX Live也应该能编译成功。}本文类可能不支持老旧的CTeX套装,但也欢迎大家使用CTeX环境尝试,如果能编译成功欢迎大家使用!CTeX套装因为WinEdt的版权问题早已无人维护,中文的LaTeX排版采用成熟的ctex宏包更为方便。文类已经加载的宏包有:ctex, geometry, fontenc, mathptmx, xkeyval, titlesec, titletoc, amsmath, amsfonts, amssymb, bm, array, longtable, booktabs, multirow, graphicx, subfig, flafter, caption, hyperref, cleverref, ulem, xcolor, fancyhdr, listings, paralist。
\subsection{选项与设置}
本文类提供了title、institute、majorclass、name、mentor、date等宏包选项,对应的分别是论文题目、所在学院、专业班级、姓名、指导教师、日期等项目。若论文题目太长,可用\textbackslash\textbackslash 使其折行,例如本文的相关的设置为

\centerline{title = 大连海事大学本科毕业设计(论文)\textbackslash\textbackslash\ \textbackslash LaTeX 文档类}

另外,定义了一个onelinetitle的命令,用于在页眉输出一行形式的题目,在重新定义onelinetitle处,只需把title中的\textbackslash\textbackslash 删除然后赋给onelinetitle即可。其他设置比较简单,可见template-example.tex文件的相关设置。

欢迎大家去海大\LaTeX\ 学习交流群讨论相关技术问题,\textbf{QQ群号码为:835453813}
\subsubsection{三级标题}
这里是三级标题的测试。
\paragraph{四级标题}
这里是四级标题的测试,但四级标题并不会出现在目录。