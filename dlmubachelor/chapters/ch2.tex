\section{公式、表格与插图}
文类使用了cleverref宏包,这使得公式、表格与插图的交叉引用非常智能化。
\subsection{公式}
公式的一个简单例子:

\cref{gougueq}是勾股定理的公式,其中$ a\text{、}b $表示两个直角边,$ c $表示斜边。
\begin{equation}
	\label{gougueq}
	a^2+b^2=c^2
\end{equation}

其代码如下:
\begin{verbatim}
	\cref{gougueq}是勾股定理的公式,其中$ a\text{、}b $表示两个直角边,$ c $表示斜边。
	\begin{equation}
		\label{gougueq}
		a^2+b^2=c^2
	\end{equation}
\end{verbatim}
\subsection{表格}
这里主要演示三线表的使用,\cref{tab1}展示了文类使用的一些宏包。
\begin{table}[htbp]
	\centering
	\caption{文类使用的部分宏包}
	\label{tab1}
	\begin{tabular}{cc}
		\toprule[1pt]
		% 我比较喜欢这样改宽度
		\hspace{1cm}宏包\hspace{1cm} & \hspace{3cm}作用\hspace{3cm} \\
		\midrule[0.5pt]
		ctex                         & 支持排版中文                 \\
		geometry                     & 调整页面布局                 \\
		titlesec                     & 标题格式设置                 \\
		titletoc                     & 目录格式设置                 \\
		amsmath                      & 多种公式环境                 \\
		amsfonts                     & 数学符号、字体               \\
		amsmb                        & 数学符号、字体               \\
		bm                           & 数学粗体                     \\
		\bottomrule[1pt]
	\end{tabular}
\end{table}

下面是有关的代码,\textbf{需要强调的是:}{\color[rgb]{1.00,0.00,0.00}一定要让label写在caption后,否则交叉引用会不正确!!!}
\begin{verbatim}
	\begin{table}[htbp]
		\centering
		\caption{文类使用的部分宏包}
		\label{tab1}
	\begin{tabular}{cc}
	\toprule[1pt]
	% 我比较喜欢这样改宽度
	\hspace{1cm}宏包\hspace{1cm}	&	\hspace{3cm}作用\hspace{3cm}	\\
			\midrule[0.5pt]
			ctex	& 	支持排版中文	\\
			geometry	&	调整页面布局	\\
			titlesec	&	标题格式设置	\\
			titletoc	&	目录格式设置	\\
			amsmath	&	多种公式环境	\\
			amsfonts	&	数学符号、字体	\\
			amsmb	&	数学符号、字体	\\
			bm	&	数学粗体	\\
			\bottomrule[1pt]
		\end{tabular}	
	\end{table}
\end{verbatim}

\subsection{图片}
文类使用了graphicx,subfig宏包来排版图片,\cref{ctanlion}是单图排版的一个简单例子。
\begin{figure}[htbp]
	\centering
	\includegraphics[width=0.6\linewidth]{ctan_lion}
	\caption{ctan小狮子}
	\label{ctanlion}
\end{figure}

其代码如下:
\begin{verbatim}
	\begin{figure}[htbp]
		\centering
		\includegraphics[width=0.6\linewidth]{ctan_lion}
		\caption{ctan小狮子}
		\label{ctanlion}
	\end{figure}
\end{verbatim}

\cref{twofig}是双图排版的一个简单案例,其中\cref{sf:ctanlion}是ctan官网小狮子,\cref{sf:colorful}是着色的小狮子。

\begin{figure}[htbp]
	\centering
	\subfloat[ctan小狮子\label{sf:ctanlion}]{
		\includegraphics[width=0.3\linewidth]{ctan_lion}
	}
	\hspace{1em}
	\subfloat[着色小狮子\label{sf:colorful}]{
		\includegraphics[width=0.3\linewidth]{colorful_lion}
	}
	\caption{双图排版}
	\label{twofig}
\end{figure}

下面是相相关代码:
\begin{verbatim}
	\begin{figure}[htbp]
		\centering
		\subfloat[ctan小狮子\label{sf:ctanlion}]{
			\includegraphics[width=0.3\linewidth]{ctan_lion}
		}
		\hspace{1em}
		\subfloat[着色小狮子\label{sf:colorful}]{
			\includegraphics[width=0.3\linewidth]{colorful_lion}
		}
		\caption{双图排版}
		\label{twofig}
	\end{figure}
\end{verbatim}

\cref{twofig2}是另一种双图排版的实现方法,\textbackslash subfloat命令缺少宽度参数,而子标题最多只能和子图一样宽,太长的话会出现折行。为了避免子标题折行,\textbackslash subfloat里再嵌套个minipage,因为后者是有宽度的\upcite{btl}。在此处页展示了参考文献上标引用方法,其命令为\verb|\upcite{btl}|;若不需要上标形式,即\cite{btl}这种格式,可直接用cite引用命令。\textbf{需要注意的是,图片的宽度要小于minipage环境的宽度才能正确排版出图片!!!}

\begin{figure}[htbp]
	\centering
	\subfloat[ctan小狮子\label{sf:ctanlion2}]{
		\begin{minipage}{5cm}
			\centering
			\includegraphics[width=4.5cm]{ctan_lion}
		\end{minipage}
	}
	\subfloat[着色小狮子\label{sf:colorful2}]{
		\begin{minipage}{5cm}
			\centering
			\includegraphics[width=4.5cm]{colorful_lion}
		\end{minipage}
	}
	\caption{另一种双图排版}\label{twofig2}
\end{figure}


下面是相关代码:
\begin{verbatim}
	\begin{figure}[htbp]
		\centering
		\subfloat[ctan小狮子\label{sf:ctanlion2}]{
			\begin{minipage}{5cm}
				\centering
				\includegraphics[width=4.5cm]{ctan_lion}
			\end{minipage}
		}
		\subfloat[着色小狮子\label{sf:colorful2}]{
			\begin{minipage}{5cm}
				\centering
				\includegraphics[width=4.5cm]{colorful_lion}
			\end{minipage}
		}
		\caption{另一种双图排版}\label{twofig2}
	\end{figure}
\end{verbatim}

